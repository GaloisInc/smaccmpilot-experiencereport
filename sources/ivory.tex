\section{Ivory: Safe C Programming}
\label{sec:ivory}

At face value, our approach sounds audacious if not ludicrous: faced with a
deadline for developing a new high-assurance autopilot system in one-and-a-half
years, start by designing a new programming language and compiler from the
ground up.

Of course, developing an EDSL is not the same as developing a stand-alone
compiler.  Much of the typical compiler tool-chain, such as the front-end
parser/lexer, is provided for free by the host language.  It took approximately
6 engineer-months to create the Ivory language and compiler, which totals about
6k lines of Haskell code.

The language we developed for generating safe embedded C code is called
\emph{Ivory}.  Ivory compiles to restricted C code suitable for embedded
programming.  Ivory shares the goal of other ``safe-C'' standalone languages and
compilers like Cyclone~\cite{cyclone} and Rust~\cite{rust}.  Our main motivation
for not using those languages is our desire for a convenient, Turing-complete,
type-safe macro-language (Haskell) to improve our productivity.

There have also been some ``safe-C'' EDSLs including Atom~\cite{atom},
Copilot~\cite{copilot}, and Feldspar~\cite{feldspar1}.  The most significant
difference between these languages and Ivory is that they are focused on pure
computations (e.g., Feldspar is a DSL for digital signal processing), and do not
provide convenient support for defining in-memory data-structures and
manipulating memory.  Ivory is designed to be a EDSL that can be used for
writing safe memory-manipulating embedded C.

Ivory also makes  contributions from a programming language
perspective, namely in its expressiveness and type-safety.  We overview each,
then present a small example, to give the reader a feel for the language.

\paragraph{Expressiveness}
Regarding the expressiveness, Ivory has a variety of useful features, including:
\begin{itemize}
  \item \emph{Memory-areas}: the ability to allocate stack-based memory and
    manipulate both local and global memory areas~\cite{memareas}.
  \item \emph{Product types}: C structs with well-typed accessors.
  \item \emph{FFI}: typed interfaces for calling arbitrary C functions.
  %% \item \emph{Module system}: managing dependencies to support C-level headers
  %%   and includes.
  \item \emph{Bit-fields}: support for typed manipulation of bit-fields and
    registers~\cite{high-level}.
\end{itemize}

We built Ivory with some limitations to simplify generating safe C programs.
Ivory does not support heap-based dynamic memory allocation (but global
variables can be defined).  C arrays are fixed-length.  There is no pointer
arithmetic.  Pointers are non-nullable.  Union types are not supported.  Unsafe
casts are not supported: casts must be to a strictly more expressive type (e.g.,
from an unsigned 8-bit integer to an unsigned 16-bit integer) or a default value
must be provided for when the cast is not valid. The most common unsafe C cast
is not possible: no void-pointer type exists in Ivory.

In Ivory, these have not been limiting factors, particularly because of the
power of using Haskell as a macro system.  For example, while arrays must be of
fixed size at C compile-time, we can define a single \emph{Haskell} function
that is polymorphic in the array size that becomes instantiated at a particular
size at each use site.  %% An array has the type

%% \begin{code}
%% Array (n::Nat) t
%% \end{code}

\lp{talk about guarantees that can't store a local ref into a global}

\paragraph{Type-checking}
Ivory's domain-specific type checking focuses on guaranteeing memory
safety and helping programmers reason about their programs' nonfunctional
behaviors more easily.

%% Our claim is that for the fragment of C generated, these type checks are
%% sufficient to guarantee generated programs are memory safe (a forthcoming paper
%% will make this argument explicit).

%% With respect to memory safety, the essential
%% guarantees necessary, for our restricted form of C, are that
%% \begin{itemize}
%%   \item null pointers are never dereferenced;
%%   \item array indexing occurs only within the memory allocated to
%%     the array;
%%   \item no dangling pointers are possible; in particular, a function should not
%%     return a pointer to memory allocated within its own stack frame.
%% \end{itemize}

In addition, Ivory programs have an \emph{effects} type associated with them,
implemented as a parameter to the Ivory monad.  There are three kinds of effects
tracked:
\begin{itemize}
  \item \emph{Allocation effects}: whether a program performs (stack-based)
    memory allocation as well as whether pointers point into global or stack memory.
  \item \emph{Return effects}: whether a program contains a \cd{return} statement.
  \item \emph{Break effects}: whether a program contains a \cd{break}
    statement.
\end{itemize}
\noindent
Allocation effects allow memory allocation to be restricted and tracked at the
type level.  For example, from a program's type alone, we can determine whether
it allocates memory on the stack, making stack usage easier to track.  More
importantly for memory-safety, allocation effects also ensure Ivory programs
contain no dangling pointers: it is a type error to return a pointer to
locally-allocated memory.

Return and break statements fundamentally affect control-flow and can result in
unexpected behavior by breaking out of the current block or returning from a
function.  For example, in a top-level while loop implementing an real-time
operating system task, there should be no break or return statements; we can
enforce this with the type system.  Tracking these effects is novel, we believe,
and particularly important in the context of an EDSL in which programs are
generated and manipulated heavily in the host language.

In an EDSL, we have at least two options for type checking: (1) write a
domain-specific type-checker \emph{in} Haskell (relying on Haskell's type-system
just for macro-language type-checking), or (2) embed the domain-specific type
checker into Haskell's type system.

We were motivated to pursue option (2) because it allows us to discover problems
sooner in the development cycle. In the case of option (1), we only find out
about problems in the program's AST during code generation. Option (2) ensures
that all macro and library code is typed correctly, independent of its use in
the generated code. We discuss the issues of finding errors early on in more
detail in Section~\ref{sec:thegood}.

When we began developing Ivory, our hypothesis  was that recent type-system
extensions to the Glasgow Haskell Compiler make it feasible to embed the
invariants necessary to ensure memory-safe C programming into the
type-system~\cite{dephaskell}. From a practical standpoint, Ivory demonstrates
just how far the type-system has come, allowing us to replicate the type safety
of compilers like Cyclone, etc.

We do not have space to adequately describe Ivory's type system; we leave that
to a forthcoming paper.  Here we will note that the embedding depends on the use
of data kinds~\cite{datakinds}, type families~\cite{typefamilies}, and rank-2
polymorphism~\cite{stmonad}.


\paragraph{Ivory example}
We present a small example of Ivory code.  The example omits many features of
the language, but should give the reader a feeling for it.

\begin{figure}
    \begin{smcode}
[ivory|
struct fooSstruct
  \{ bar :: Stored Uint8
  ; baz :: Array 10 (Stored Sint16)
  \}
|]

setBaz :: Def ([Ref Global (Struct "fooStruct"), Sint16] :-> ())
setBaz = proc "setBaz" $ \textbackslash{}ref val -> body (prgm ref val)

prgm :: Ref Global (Struct "fooStruct") -> Sint16 -> Ivory eff ()
prgm ref val = arrayMap $ \textbackslash{}ix ->
                 store ((ref \(\sim\)> baz) ! ix) val

\(\noindent\rule{6cm}{0.4pt}\)

// foo_source.c
#include "foo_module.h"

void setBaz(struct fooStruct* n_var0, int16_t n_var1) \{
  for ( int32_t n_ix0 = (int32_t) 0
      ; n_ix0 <= (int32_t) 9
      ; n_ix0++ ) \{
      n_var0->baz[n_ix0] = n_var1;
  \}
\}

// foo_module.h
struct fooStruct \{
    uint8_t bar;
    int16_t baz[10U];
\};

void setBaz(struct fooStruct* n_var0, int16_t n_var1);
\end{smcode}
  \caption{Example Ivory module definition}
  \label{fig:module}
\end{figure}
%% $

Consider Figure~\ref{fig:module}, in which an Ivory program is shown, as well as
the corresponding generated C sources and headers (making a few syntactic
changes to the C for readability, not relevant to the example).

First, we define a struct (or product type) using a quasiquoter that is part of
the Ivory language.  The Ivory code generated by Template
Haskell~\cite{th} constructs a struct definition containing two fields
consisting of an unsigned byte and an array of 10 signed 16-bit integers.
Template Haskell also constructs a new type-level literal, \cd{"fooStruct"},
that is unique to the defined struct.  The \cd{Stored} type constructor
signifies that the value is allocated in-memory~\cite{memareas}.  The \cd{Array}
type constructor takes a type-level natural number as a parameter (available as
a Glasgow Haskell Compiler extension) to fix the size of an array.

A procedure, corresponding to a C function, has a type of the form
\begin{code}
Def (params :-> out)
\end{code}
\noindent
where \cd{params} are the procedure's parameter types and \cd{out} is its return
type.  The procedure \cd{setBaz} takes two arguments and its return type is
unit, corresponding to the \cd{void} type in C.  The types of the procedure's
arguments are types in a type-level list: the first argument is a
\emph{reference}, a non-null pointer by construction, to a struct, and the
second argument is a signed 16-bit integer.  The \cd{Ref} type constructor takes
a \emph{scope} type and a memory-area type.  The scope type denotes either
stack-allocated scope, or global (and statically allocated) scope.  In the
example, we expect the reference to be to a global.

%% \begin{code}
%% Ref s (Struct "foo\_struct")
%% \end{code}
%% \noindent

Procedures are defined with the \cd{proc} operator that takes a string,
corresponding to the name of the function that will be generated in C, and a
function from the procedures arguments to its body.  The body of the function is
an Ivory program that sets each element in the \cd{baz} field of the struct with
the value \cd{val} passed to it, leaving the \cd{bar} field unchanged.

Following~\cite{memareas}, Ivory guarantees memory-safe array access in the type
system since array lengths are statically known.  Ivory provides an \cd{arrayMap}
operator that applies a function to each valid index into the array.  The
function applied in this case is a \cd{store} operation that takes a reference
to a memory area, a value, and stores the value in the area.  It is a type-error
if the value's type and memory-area's type do not match.

The operation (\cd{ref \mytilde> baz}) takes the struct reference and
returns a reference to the \cd{baz} field.  The bang (\cd{!}) operator takes a
reference to an array, an index, and returns a reference to the value at that
index.  The safety of indexing is maintained since the operator has the type
\begin{code}
(!) :: Ref s (Array len area) -> Ix len -> Ref s area
\end{code}
\noindent
tying the length of the array to the maximum index.  For example, an index type
(\cd{Ix 10}) supports index values from 0 to 9.

The example only shows a small part of Ivory's language and does not exhibit
some of its additional features to prevent unsafe programs.  For example, if
\cd{setBaz} had allocated stack memory and created a reference to it, then tried
to return the reference (creating a dangling pointer), it would result in a type
error.

Additionally, for application-specific properties that cannot be type-checked,
Ivory permits the insertion of assertions, assumptions on arguments, and
requirements on return values.  Ivory also automatically inserts checks for
arithmetic underflow/overflow and division-by-zero.  All these checks are useful
during testing and we have used them to assist with static analysis and
model-checking the generated C.


%% \begin{code}
%% (!) :: Ref s (Array len area) -> Ix len -> Ref s area
%% \end{code}

%% Now we describe the procedure \cd{setBaz}.  In defining the procedure's
%% implementation, the lambda-bound variable \cd{ref}'s type matches the type in
%% the type-level list in \cd{setBaz}'s type declaration.  Before the body of a
%% procedure, the user may add expression-level preconditions on inputs and
%% postconditions on return values. (These are not shown in the current example).

%% The body of the procedure is defined by \cd{prgm}.  Its return type is an Ivory
%% expression, returning unit.  The Ivory monad also contains an effects parameter
%% that tracks effects, as described above, using a type-level record.



%% To access the \cd{baz} field of the struct, Ivory has a field accessor,
%% reminiscent of C's \cd{->} operator
%% \begin{code}
%% (\(\sim\)>) :: Ref s (Struct symbol)
%%       -> Label symbol field -> Ref s field
%% \end{code}
%% \noindent
%% eliding type class and data-kind constraints.  Note, however, that gives
%% \cd{\mytilde{}>} returns a \emph{reference} to the field, not the field value itself.

%% The \cd{bar} field of the struct pointed to is updated with 1.  Updating
%% \cd{baz} is slightly more complicated since it contains an array. In
%% the example, the 0th index of the array in the \cd{baz} field is updated with 2
%% (the other indexes are unmodified).


%% Array indexing is guaranteed to be safe by the type system.  Like arrays,
%% indexes into arrays have types that are parameterized by a type-level natural
%% number.  An index type \cd{Ix n} only supports indexes from \cd{0} to \cd{n-1}.
%% Eliding type-class and data-kind constraints for the sake of presentation, the
%% type of the index operator ensures that the array length and index size match:
%% \begin{code}
%% (!) :: Ref s (Array len area) -> Ix len -> Ref s area
%% \end{code}

%% The final interface we describe in this example if our EDSL ``module''
%% interface.  A \cd{Module} contains all of the top-level procedures and struct
%% definitions packaged to be passed to the EDSL compiler for optimization and code
%% generation.  The module \cd{foo\_module} contains the definition for the struct
%% and the implementation of the function.  (Because structure names are type-level
%% strings which are not \cd{*}-kinded, a proxy type is given to turn an
%% arbitrary-kinded type into a \cd{*}-kinded type, which can be passed to a
%% function.)

%% We discuss drawbacks of this module system design in Section~\ref{sec:thegood}.


















%% we define a type-level record to track effects.  We do not have
%% space to describe the full embedded type system, but hope to do so in a
%% forthcoming paper describing the Ivory language.


%% \paragraph{Effects}
%%   Using the data kind extension to lift data types to define kinds and
%% types, we define a kind \cd{RetEff} with the type \cd{Ret r} denoting that an
%% Ivory code block contains a return statement, and if so, the type of the value
%% returned.  If there is no return statement, the associated type is \cd{NoRet}.
%% We similarly define an allocation kind containing types that denote whether
%% there is local or global memory allocation as well as the type of the value
%% allocated.  Using type families, we define rewrite rules that are type-level
%% record selectors that take a type of kind \cd{Effects} and returns the field (a
%% type) of interest.

%% There is an additional effect tracking the use of
%% \cd{break} statements, which may optionally appear only inside loop bodies.
%% So, some combinators in Ivory can reset effects:
%% \cd{break} is not permitted in procedure bodies, but introducing a loop
%% combinator allows \cd{break} statements to be used in that body.

%% The approach of faking
%% type-level records allows us to encode an arbitrary number of effects in a
%% single type.



%% The basic syntax of Ivory is the statement, which make up all procedures.
%% Ivory statements are effectful, we used a traditional a monadic interface.
%% Internally, the Ivory monad is a writer transformer (to record
%% statements) over the state monad (to generate fresh names). As discussed above,
%% the Ivory monad type is parameterized by an effects to restrict effects in
%% statements. An example given in Figure~\ref{fig:tower-ex} is explained below.


%% \begin{figure}
%%     \begin{smcode}
%%     \end{smcode}
%%   \caption{Ivory Struct type definition}
%%   \label{fig:structqq}
%% \end{figure}

%% \lp{TODO squigglies}



%% The module system was one major language feature Ivory could not inherit from
%% it's host language. Because Haskell does not have any mechanism for reflection
%% of module contents, the Ivory programmer must create modules explicitly.

%% In Ivory, procedures and global memory areas are ordinary Haskell values. These,
%% along with dependencies, type definitions, and foreign function headers, must be
%% listed as the contents of an Ivory module. We use a monadic interface for
%% building modules, but the only action supported by this \cd{ModuleM} monad
%% is writing: all values have type \cd{ModuleM ()}; essentially it is the same as
%% building a list, but with slightly more convenient syntax. The example module
%% in Figure~\ref{fig:module} shows a module \cd{foo\_module} which contains:
%% \begin{itemize}
%%   \item a type definition of \cd{Struct "foo"}, from Figure~\ref{fig:structqq}
%%   \item a function \cd{setBaz}
%%   \item a global memory area \cd{foo\_global}.
%% \end{itemize}
%% \lp{TODO make sure lists and compact lists are the same}

%% Ivory modules are translated to C as a single module at a time. Due to this
%% restriction, the compiler cannot guarantee that each module fully describes all
%% of its type, function, \& memory area dependencies, because the compilation
%% cannot assume any knowledge of the contents of dependencies. This is for good
%% reason: since dependencies may be foreign (non-Ivory generated) header files,
%% they may not be possible to describe in the Ivory type system.

%% We did not pursue whole-program compilation, where an entire set of
%% interdependent modules could be checked against each other for consistency.
%% However, it is possible that changing our approach to the module system would
%% permit us to catch errors earlier without undue burden to the programmer. We
%% plan to investigate this problem further in the future.

%% \begin{code}
%% foo3 :: Def (`[Ref s (Array 10 (Stored Uint32))] :-> ())
%% foo3 = proc "foo" $ \textbackslash{}arr -> body $ do
%%   arrayMap $ \textbackslash{}ix -> do
%%     v <- deref (arr ! ix)
%%     store (arr ! ix) (v+1)
%% \end{code}
%% %% $


%% Memory-allocated types have a distinct kind.  The benefit of doing so is that
%% type classes with methods operating on memory areas are defined closed under .
%% The type of all data that can be allocated (either globally or locally) has kind
%% \cd{Area k}:
%% \begin{code}
%% data Area k
%%   = Array Nat (Area k)
%%   | Struct Symbol
%%   | ...
%% \end{code}
%% \noindent
%% We have elided a few additional memory area types.  Note to that the data type
%% is used only to define a kind and its types~\cite{}; we do not use \cd{Array},
%% etc. as value constructors.  The type constructor for an array type
%% takes a type-level natural number (from \cd{GHC.TypeLits}) representing its
%% length at the type-level and the type of the values contained in the array.
%% Similarly, the type constructor for a struct type takes a type-level string that
%% names the struct definition.


%% \begin{itemize}
%% \item Brief overview of Ivory.
%% \item Note that types are embedded in GHC type-checker.
%% \item Philosophy: push errors higher into the design cycle (C runtime to C
%%   compile time to DSL runtime to typechecking time).
%% \item Insert checks into compiler.  Easy with EDSL  (blog post).
%% \end{itemize}


%% \begin{compactenum}
%%   \item Haskell type-checking (Haskell compilation)
%%   \item Domain-specific type-checking
%%   \item C compilation
%%   \item Static analysis
%%   \item Execution
%% \end{compactenum}

