%-----------------------------------------------------------------------------
%
%               Template for sigplanconf LaTeX Class
%
% Name:         sigplanconf-template.tex
%
% Purpose:      A template for sigplanconf.cls, which is a LaTeX 2e class
%               file for SIGPLAN conference proceedings.
%
% Guide:        Refer to "Author's Guide to the ACM SIGPLAN Class,"
%               sigplanconf-guide.pdf
%
% Author:       Paul C. Anagnostopoulos
%               Windfall Software
%               978 371-2316
%               paul@windfall.com
%
% Created:      15 February 2005
%
%-----------------------------------------------------------------------------


\documentclass{sigplanconf}

% The following \documentclass options may be useful:

% preprint      Remove this option only once the paper is in final form.
% 10pt          To set in 10-point type instead of 9-point.
% 11pt          To set in 11-point type instead of 9-point.
% authoryear    To obtain author/year citation style instead of numeric.

\usepackage{ifthen}
\usepackage{color}
\usepackage{framed}
\usepackage{paralist}
\usepackage{mathtools}
\usepackage{alltt}
\usepackage{textcomp}
\usepackage{fixltx2e}
\usepackage{url}

\newboolean{td}
  \setboolean{td}{false} % modify here
  \ifthenelse{\boolean{td}}
             {\newcommand{\ph}[1]{\textcolor{blue}{PH: #1}}
               \newcommand{\lp}[1]{\textcolor{blue}{LP: #1}}
             }
             {\newcommand{\ph}[1]{}
               \newcommand{\lp}[1]{}
             }

\newcommand{\mytilde}{\raise.17ex\hbox{$\scriptstyle\mathtt{\sim}$}}
\newenvironment{code}{\begin{alltt}\footnotesize}{\end{alltt}}
\newenvironment{smcode}{\begin{alltt}\scriptsize}{\end{alltt}}

%% \newenvironment{cols}{\begin{tabular}{m{3.6cm}|m{3.6cm}} &
%%     \\\hline}{\end{tabular}}

\newcommand{\cd}[1]{\texttt{#1}}

\begin{document}

\special{papersize=8.5in,11in}
\setlength{\pdfpageheight}{\paperheight}
\setlength{\pdfpagewidth}{\paperwidth}

\conferenceinfo{ICFP'14}{September 01 -- 06, 2014, Gothenburg, Sweden}
\copyrightyear{2014}
\copyrightdata{978-1-4503-2873-9/14/09}
\doi{2628136.2628146}

% Uncomment one of the following two, if you are not going for the 
% traditional copyright transfer agreement.

%\exclusivelicense                % ACM gets exclusive license to publish, 
                                  % you retain copyright

\permissiontopublish             % ACM gets nonexclusive license to publish
                                  % (paid open-access papers, 
                                  % short abstracts)

\titlebanner{Under submission}        % These are ignored unless
\preprintfooter{short description of paper}   % 'preprint' option specified.

\title{Building Embedded Systems with Embedded DSLs}
\subtitle{(Experience Report)}

\authorinfo{Patrick C. Hickey \and Lee Pike \and Trevor Elliott \and James Bielman
           \and John Launchbury}
           {Galois, Inc.}
           {\{pat, leepike, trevor, jamesjb, john\}@galois.com}
%% \authorinfo{Name2\and Name3}
%%            {Affiliation2/3}
%%            {Email2/3}

\maketitle

\begin{abstract}
We report on our experiences in synthesizing a fully-featured autopilot from
embedded domain-specific languages (EDSLs) hosted in Haskell.  The autopilot is
approximately 50k lines of C code generated from 10k lines of EDSL code and
includes control laws, mode logic, encrypted communications system, and device
drivers.  The autopilot was built in less than two engineer years.
This is the story of how EDSLs provided the
productivity and safety gains to do large-scale low-level embedded programming
and lessons we learned in doing so.
\end{abstract}

\category{D.3.2}{Language Classifications}{Applicative (functional) languages}

% general terms are not compulsory anymore,
% you may leave them out
%% \terms
%% term1, term2

\keywords
Embedded Domain Specific Languages, Embedded Systems

\section{Introduction}

Embedded programming involves the lowest levels of abstraction.  Most development
is in low-level languages, like C or assembly, and programs interact intimately
with the hardware.  Embedded domain-specific languages (EDSLs) are in
some sense at the other end of the software spectrum: they are often embedded
(in a different sense of the word!) in high-level programming languages such as
Haskell or ML, and are used to lift the abstraction level.

That said, there is no reason in-principle why EDSLs cannot be used for embedded
programming; this report is about our experience in building new EDSLs for
embedded programming and the benefits and difficulties in using them.  Our
experiences are based on building an autopilot system called SMACCMPilot from
our EDSLs.  The breadth and scope of the project sets it apart.  The autopilot
software is not just an application, but a complete embedded system that
includes not just the core flight control algorithms, but also device drivers,
encrypted network stack, mode logic, and concurrency and task management.  It is
one of the largest (open-source) embedded systems projects developed using the
EDSL approach.

Our story is a largely positive one: we developed the languages and their (EDSL)
compilers from scratch in approximately 14 engineer-months then we used them to
build the SMACCMPilot hardware support and application in another 22
engineer-months.  We achieved a dramatic increase in productivity as well as
code quality.  By construction, the generated C excludes large classes of errors
and undefined behaviors.

Our goal in this paper is to summarize some of our lessons-learned.  While we
use specific examples, the lessons apply more generally to industrially-large
embedded system design in EDSLs.  Our target audience includes both researchers
developing new EDSLs for low-level programming as well as practitioners
considering using EDSLs.

All of the EDSLs described herein, as well as SMACCMPilot itself, are
open-source software. Further documentation is available at \url{smaccmpilot.org}.


%% \lp{Indeed, a number of researchers have explored EDSLs for generating embedded
%%   C code, such as Atom~cite{}, Feldspar~\cite{}, Copilot~\cite{}, \lp{more}.
%%   However, these languages are still at the application level }

%% \lp{In this experience report, we describe lessons learned in the use of EDSLs to
%% build a complex embedded system: a secure autopilot called
%% \emph{SMACCMPilot}.\footnote{\emph{SMACCM} is an acronym for Secure
%%   Mathematically-Assured Composition of Control Models.}
%% }



\section{Ivory: Safe C Programming}
\label{sec:ivory}

At face value, our approach sounds audacious if not ludicrous: faced with a
deadline for developing a new high-assurance autopilot system in one-and-a-half
years, start by designing a new programming language and compiler from the
ground up.

Of course, developing an EDSL is not the same as developing a stand-alone
compiler.  Much of the typical compiler tool-chain, such as the front-end
parser/lexer, is provided for free by the host language.  Furthermore, we were
confident that time spent in writing the EDSL would be made up by developing in
a high-level functional language with a rich set of existing libraries. It took
approximately 6 engineer-months to create the language and compiler, and
total of about 6k lines of Haskell code.

The language we developed for generating safe embedded C code is called
\emph{Ivory}.  Ivory shares the goal of other ``safe-C'' standalone languages
and compilers like Cyclone~\cite{cyclone} and Rust~\cite{rust}.  Our main
motivation for not using these languages is to have the benefits of an EDSL and
particularly our desire for a convenient, Turing-complete, type-safe
macro-language (Haskell) to improve our productivity.

There have also been some ``safe-C'' EDSLs including Atom~\cite{atom},
Copilot~\cite{copilot}, and FeldSpar~\cite{feldspar1}.  The most significant
difference between these languages and Ivory is that they are focused on pure
computations and do not provide convenient support for defining in-memory
data-structures and manipulating memory.  Ivory is designed to be a
general-purpose embedded C EDSL that can be used for memory-manipulation
intensive programs, such as device drivers.

The essential contributions of Ivory from a programming language perspective are
its expressiveness and our approach to type-safety.

\paragraph{Expressiveness}
Regarding the expressiveness, Ivory has a variety of useful features, including:
\begin{itemize}
  \item \emph{Memory-areas}: the ability to allocate stack-based memory and
    manipulate both local and global memory areas~\cite{memareas}.
  \item \emph{Product types}: C structs with well-typed accessors.
  \item \emph{FFI}: typed interfaces for calling arbitrary C functions.
  %% \item \emph{Module system}: managing dependencies to support C-level headers
  %%   and includes.
  \item \emph{Bit-fields}: support for typed manipulation of bit-fields and
    registers~\cite{high-level}.
\end{itemize}

We built Ivory with some limitations in order to  ensure safe C programs are
generated.  Ivory does not support heap-based dynamic memory allocation (but
global variables can be defined).  C arrays are fixed-length.  There is no
pointer arithmetic.  Pointers are non-nullable.  Union types are not supported.
Unsafe casts are not supported: casts must be to a strictly more expressive type
(e.g., from an unsigned 8-bit integer to an unsigned 16-bit integer) or a
default value must be provided for when the cast is not valid. The most common
unsafe C cast is not possible: no void-pointer type exists in Ivory.

In Ivory, these have not been limiting factors, particularly because of the
power of using Haskell as a macro system.  For example, while arrays must be of
fixed size at C compile-time, we can define a single \emph{Haskell} function
that is polymorphic in the array size that becomes instantiated at a particular
size at each use site.  %% An array has the type

%% \begin{code}
%% Array (n::Nat) t
%% \end{code}

\paragraph{Type-checking}
Ivory's domain-specific type checking focuses on guaranteeing memory
safety and helping programmers reason about their programs' nonfunctional
behaviors more easily.

%% Our claim is that for the fragment of C generated, these type checks are
%% sufficient to guarantee generated programs are memory safe (a forthcoming paper
%% will make this argument explicit).

%% With respect to memory safety, the essential
%% guarantees necessary, for our restricted form of C, are that
%% \begin{itemize}
%%   \item null pointers are never dereferenced;
%%   \item array indexing occurs only within the memory allocated to
%%     the array;
%%   \item no dangling pointers are possible; in particular, a function should not
%%     return a pointer to memory allocated within its own stack frame.
%% \end{itemize}

In addition, Ivory programs have an \emph{effects} type associated with them.
There are three kinds of effects tracked:
\begin{itemize}
  \item \emph{Allocation effects}: whether a program performs (stack-based)
    memory allocation as well as whether pointers point into global or stack memory.
  \item \emph{Return effects}: whether a program contains a \cd{return} statement.
  \item \emph{Break effects}: whether a program contains a \cd{break}
    statement.
\end{itemize}
\noindent
Tracking allocation effects allows memory allocation to be restricted and
tracked.  Return and break statements fundamentally effect control-flow and can
result in unexpected behavior by breaking out of the current block or returning
from a function.  Such effects are particularly dangerous in the context of an
EDSL in which programs are generated and manipulated heavily in the host
language.  Combining allocation and return effects, the type system guarantees
that a pointer into the stack cannot be returned by a function (which is indeed
memory-unsafe).

In an EDSL, we have at least two options for type checking: (1) write a
domain-specific type-checker \emph{in} Haskell (relying on Haskell's type-system
just for macro-language type-checking), or (2) embed the domain-specific type
checker into Haskell's type system.

For a large system such as SMACCMPilot, the development times are substantial.
Therefore, our goal is to learn about program errors early, motivating us to
pursue option (2), to detect errors during type-checking, which is fast and
modular, rather than after Haskell compilation, when executing the EDSL programs
to generate C code.  (We discuss the issues of finding errors early on in more
detail in Section~\ref{sec:thegood}.)

Our hypothesis when developing Ivory was that recent type-system extensions to
the version of Haskell implemented by the Glasgow Haskell Compiler make it
feasible to embed the invariants necessary to ensure memory-safe C programming
into the type system~\cite{dephaskell}.  From a practical standpoint, doing so
demonstrates just how far the type-system has come, allowing us to replicate the
type safety of compilers like Cyclone, etc.

We do not have space to adequately describe Ivory's type system; we leave that
to a forthcoming paper.  Here we will note that the embedding depends on the use
of data kinds~\cite{datakinds}, type families~\cite{typefamilies}, and rank-2
polymorphism~\cite{stmonad}.

%% we define a type-level record to track effects.  We do not have
%% space to describe the full embedded type system, but hope to do so in a
%% forthcoming paper describing the Ivory language.


%% \paragraph{Effects}
%%   Using the data kind extension to lift data types to define kinds and
%% types, we define a kind \cd{RetEff} with the type \cd{Ret r} denoting that an
%% Ivory code block contains a return statement, and if so, the type of the value
%% returned.  If there is no return statement, the associated type is \cd{NoRet}.
%% We similarly define an allocation kind containing types that denote whether
%% there is local or global memory allocation as well as the type of the value
%% allocated.  Using type families, we define rewrite rules that are type-level
%% record selectors that take a type of kind \cd{Effects} and returns the field (a
%% type) of interest.

%% There is an additional effect tracking the use of
%% \cd{break} statements, which may optionally appear only inside loop bodies.
%% So, some combinators in Ivory can reset effects:
%% \cd{break} is not permitted in procedure bodies, but introducing a loop
%% combinator allows \cd{break} statements to be used in that body.

%% The approach of faking
%% type-level records allows us to encode an arbitrary number of effects in a
%% single type.


\paragraph{Programmer Interface}
Ivory has different ``inferfaces'' for writing programs, defining procedures to
encapsulate programs (and that are compiled to C functions), defining data
structures, and for constructing modules.  We cannot cover all of these here, so
we focus on the most central, explaining them by way of an example.

\begin{figure}
    \begin{smcode}
[ivory|
struct foo\_struct
  \{ bar :: Stored Uint8
  ; baz :: Array 10 (Stored Sint16)
  \}
|]

foo\_init :: Def(`[Ref s (Struct "foo\_struct")]:-> ())
foo\_init = proc "foo\_init" $ \textbackslash{}f\_ref -> body (prgm f\_ref)

prgm :: Ref s (Struct "foo_struct") -> Ivory eff ()
prgm f_ref = do
  store  (f_ref \(\sim\)> bar)      1
  store ((f_ref \(\sim\)> baz) ! 0) 2

foo\_module :: Module
foo\_module = package "foo\_module" $ do
  defStruct (Proxy :: Proxy "foo_struct")
  incl foo\_init
    \end{smcode}
  \caption{Example Ivory module definition}
  \label{fig:module}
\end{figure}
%% $

Consider Figure~\ref{fig:module}.  First, we define a struct (or product type)
using a quasiquoter.  Struct \cd{foo\_struct} contains two fields
consisting of an unsigned byte and an array of 10 signed 16-bit integers.  The
\cd{Stored} type constructor signifies that the value is allocated
in-memory~\cite{memareas}.  The \cd{Array} type constructor takes a type-level
natural number as a parameter (implemented as a Glasgow Haskell Compiler
extension) to fix the size of an array.

The quasiquoter uses Template Haskell~\cite{th} to generate
typeclass instances for the defined struct as well as top-level values that are
field accessors.  For example, for the \cd{bar} field, a label
\begin{code}
bar :: Label "foo\_struct" (Stored Uint8),
bar = Label "bar"
\end{code}
\noindent
is generated.  We describe the use of labels shortly.

A procedure, corresponding to a C function, has a type of the form
\cd{Def (params :-> out)}
where \cd{params} are the procedure's parameter types and \cd{out} is its return
type.  The procedure \cd{foo\_init} has a single input and its return type is
unit, corresponding to the \cd{void} type in C.  The types of the procedure's
inputs are types in a type-level list (the (\cd{`}) promotes the list value
constructor to a type constructor~\cite{datakinds}).  The type
\cd{Ref s area}
is the type of a \emph{reference}, or pointer guaranteed to be non-null by
construction.  A reference type constructor takes a \emph{scope} type and a
memory-area type.  The scope type denotes stack-allocated scope, or global and
statically allocated scope.  In the example, the parameter to \cd{foo\_init} can
be a reference to either scope.  The memory area pointed to be the reference
has the type of the struct defined above.

Now we describe the procedure \cd{foo\_init}.  The function \cd{proc} takes a
string corresponding to the name of the function that will be generated in C,
then a function from the procedures arguments to its body.  The lambda-bound
variable \cd{f\_ref}'s type is
\begin{code}
Ref s (Struct "foo\_struct")
\end{code}
\noindent
Before the body of a procedure, the user may add expression-level preconditions
on inputs and postconditions on return values. (These are not shown in the
current example).

The body of the procedure is defined by \cd{prgm}.  Its return type is an Ivory
expression, returning unit.  The Ivory monad also contains an effects parameter
that tracks effects, as described above, using a type-level record.

\newcommand{\mytilde}{\raise.17ex\hbox{$\scriptstyle\mathtt{\sim}$}}

Two statements implement the program, both of which update memory areas.  The
\cd{store} operator takes a reference and a value, and stores the value in the
the memory area pointed to by the reference.  The struct accessor operator,
reminiscent of C's \cd{->} operator, has the type
\begin{code}
(\(\sim\)>) :: Ref s (Struct symbol)
      -> Label symbol field -> Ref s field
\end{code}
\noindent
again eliding type class and data-kind constraints.  Note, however, that
gives \cd{\mytilde{}>} gives a reference to the field, not the field value.

The \cd{bar} field of the struct pointed to is updated with 1.  Updating
\cd{baz} is slightly more complicated since it contains an array. In
the example, the 0th index of the array in the \cd{baz} field is updated with 2
(the other indexes are unmodified).

Array indexing is guaranteed to be safe by the type system.  Like arrays,
indexes into arrays have types that are parameterized by a type-level natural
number.  An index type \cd{Ix n} only supports indexes from \cd{0} to \cd{n-1}.
Eliding type-class and data-kind constraints for the sake of presentation, the
type of the index operator ensures that the array length and index size match:
\begin{code}
(!) :: Ref s (Array len area) -> Ix len -> Ref s area
\end{code}

The final interface we describe in this example if our module interface.
A \cd{Module} contains all of the top-level procedures and struct definitions to
be sent to the compiler.  The module \cd{foo\_module} contains the definition
for the struct and the implementation of the function.  Because structure names
are type-level strings which are not \cd{*}-kinded, a proxy type is given to
turn an arbitrary-kinded type into a \cd{*}-kinded type (which can be passed to
a function). We discuss drawbacks of this module system design in
Section~\ref{sec:thegood}.

%% The basic syntax of Ivory is the statement, which make up all procedures.
%% Ivory statements are effectful, we used a traditional a monadic interface.
%% Internally, the Ivory monad is a writer transformer (to record
%% statements) over the state monad (to generate fresh names). As discussed above,
%% the Ivory monad type is parameterized by an effects to restrict effects in
%% statements. An example given in Figure~\ref{fig:tower-ex} is explained below.


%% \begin{figure}
%%     \begin{smcode}
%%     \end{smcode}
%%   \caption{Ivory Struct type definition}
%%   \label{fig:structqq}
%% \end{figure}

%% \lp{TODO squigglies}



%% The module system was one major language feature Ivory could not inherit from
%% it's host language. Because Haskell does not have any mechanism for reflection
%% of module contents, the Ivory programmer must create modules explicitly.

%% In Ivory, procedures and global memory areas are ordinary Haskell values. These,
%% along with dependencies, type definitions, and foreign function headers, must be
%% listed as the contents of an Ivory module. We use a monadic interface for
%% building modules, but the only action supported by this \cd{ModuleM} monad
%% is writing: all values have type \cd{ModuleM ()}; essentially it is the same as
%% building a list, but with slightly more convenient syntax. The example module
%% in Figure~\ref{fig:module} shows a module \cd{foo\_module} which contains:
%% \begin{itemize}
%%   \item a type definition of \cd{Struct "foo"}, from Figure~\ref{fig:structqq}
%%   \item a function \cd{foo\_init}
%%   \item a global memory area \cd{foo\_global}.
%% \end{itemize}
%% \lp{TODO make sure lists and compact lists are the same}

%% Ivory modules are translated to C as a single module at a time. Due to this
%% restriction, the compiler cannot guarantee that each module fully describes all
%% of its type, function, \& memory area dependencies, because the compilation
%% cannot assume any knowledge of the contents of dependencies. This is for good
%% reason: since dependencies may be foreign (non-Ivory generated) header files,
%% they may not be possible to describe in the Ivory type system.

%% We did not pursue whole-program compilation, where an entire set of
%% interdependent modules could be checked against each other for consistency.
%% However, it is possible that changing our approach to the module system would
%% permit us to catch errors earlier without undue burden to the programmer. We
%% plan to investigate this problem further in the future.

%% \begin{code}
%% foo3 :: Def (`[Ref s (Array 10 (Stored Uint32))] :-> ())
%% foo3 = proc "foo" $ \textbackslash{}arr -> body $ do
%%   arrayMap $ \textbackslash{}ix -> do
%%     v <- deref (arr ! ix)
%%     store (arr ! ix) (v+1)
%% \end{code}
%% %% $


%% Memory-allocated types have a distinct kind.  The benefit of doing so is that
%% type classes with methods operating on memory areas are defined closed under .
%% The type of all data that can be allocated (either globally or locally) has kind
%% \cd{Area k}:
%% \begin{code}
%% data Area k
%%   = Array Nat (Area k)
%%   | Struct Symbol
%%   | ...
%% \end{code}
%% \noindent
%% We have elided a few additional memory area types.  Note to that the data type
%% is used only to define a kind and its types~\cite{}; we do not use \cd{Array},
%% etc. as value constructors.  The type constructor for an array type
%% takes a type-level natural number (from \cd{GHC.TypeLits}) representing its
%% length at the type-level and the type of the values contained in the array.
%% Similarly, the type constructor for a struct type takes a type-level string that
%% names the struct definition.


%% \begin{itemize}
%% \item Brief overview of Ivory.
%% \item Note that types are embedded in GHC type-checker.
%% \item Philosophy: push errors higher into the design cycle (C runtime to C
%%   compile time to DSL runtime to typechecking time).
%% \item Insert checks into compiler.  Easy with EDSL  (blog post).
%% \end{itemize}


%% \begin{compactenum}
%%   \item Haskell type-checking (Haskell compilation)
%%   \item Domain-specific type-checking
%%   \item C compilation
%%   \item Static analysis
%%   \item Execution
%% \end{compactenum}


\section{Tower: from Functions to Architectures}
\label{sec:tower}

\begin{figure*}[hbt!]
  \begin{tabular}{p{0.27\textwidth}|p{0.33\textwidth}|p{0.4\textwidth}}
    \begin{smcode}
blinkTower :: Tower ()
blinkTower = do
  (tx,rx) <- channel
  task "blink" (blinkTask tx)
  task "lightswitch" $
    onChannel rx $
      \textbackslash{}lit -> do
        ifte_ lit (turnOn light)
                  (turnOff light)
    \end{smcode}
&
    \begin{smcode}
blinkTask :: ChannelSource (Stored IBool)
      -> Task ()
blinkTask chan = do
  tx  <- withChannelEmitter chan
  res <- taskLocal
  onPeriod period $ \textbackslash{}now -> do
    res <- call blinkFromTime now
    emit_ tx res
  where period = Milliseconds 100
    \end{smcode}  %%$
&
    \raisebox{-\height}{\includegraphics[width=6.5cm]{figures/tower-example}}
  \end{tabular}
  \caption{Tower (Col. 2), Task (Column. 1), Graphviz output (Col. 3)}
  \label{fig:tower-ex}
\end{figure*}


In many embedded systems, programmers produce an entire system of software
that interacts with multiple input and output peripherals concurrently using a
real-time operating system (RTOS). Typical RTOSes provide just a few low-level
locking and signaling primitives for scheduling. Since microcontrollers do not
have the virtual memory managment units (MMUs) found on larger processors, the
RTOS kernel cannot protect any system memory against badly behaved user code.
These restrictions put significant burden on programmers: they must ensure all
tasks, and all communication between tasks, are implemented correctly.

During our initial development of SMACCMPilot, we found ourselves writing
high-quality C functions in Ivory, which guarantees memory-safety of the
generated code. But whenever we needed
``glue code'' to implement inter-process communication, initialize
data-structures, read the system clock, lock the processor, etc., we were forced
to abandon our well-typed world and tediously use C directly via Ivory's foreign
function interface.  Furthermore, the hand-written C is OS-specific, meaning it
would have to be rewritten for any OS port.

\paragraph{Extending Ivory}
The hand-written glue code was ruining both our productivity and our assurance
story. We wanted a language to describe the structure of the glue code that
would generate it for us.
Our key insight was that such an EDSL could be built as a macro over Ivory,
using Ivory's code-generation facilities, without losing anything.

From these ideas, the Tower EDSL was born. You can think of Tower as an
extension to the Ivory language, designed to deal with the specific concerns of
multithreaded software architecture. Tower still allows the programmer to use
all the low-level power of Ivory for general programming, but uses a separate
language for describing tasks and the connections between them.  This is one of
the great productivity features of working with EDSLs: if you discover the
language you are using is difficult, tricky, or unsafe for solving a particular
problem, you can easily extend that language with a library without modifying
the compiler.

In Tower, one specifies tasks and communication channels, and the Tower compiler generate correct
Ivory implementations, as well as architecture description artifacts. Tower
hides the dangerous low-level scheduling primitives from the user, and keeps
type information for channels (i.e., the datatype of the channel message),
expressed as Ivory types, in the Haskell type system.

Tower allows the programmer to describe a static graph of channels and tasks.
For the intended use case in high assurance systems, a static configuration of
channels and tasks makes it easy to reason about memory requirements, and
permits the system to be analyzed for schedulability.

\paragraph{Multiple interpreters}

In the Tower front end, the programmer specifies a system that can be compiled
to multiple artifacts.

Tower is designed to support different operating systems via a swappable
backend. Since all code that touches operating system primitives is generated by
Tower, it is easy for the user to specify a system and compile it for
different operating systems. Tower supports both the open-soure
FreeRTOS\cite{freertos} as well as the formally-verified
eChronos RTOS\cite{echronos} developed by NICTA.

Tower also has a backend which generates a system description in the Architecture
Analysis and Design Language (AADL)~\cite{SAE:AADL}. We also built a backend for
the Graphviz dot language.  These output formats make it possible to visualize,
analyze, and automatically check properties about the system.

We were pleased by the productivity improvements and correctness guarantees the
Tower language provided. In all, it took about 4 engineer-months to build Tower,
and a total of about 3000 lines of Haskell code.


\paragraph{Tower example}
\label{sec:examples}

In Figure~\ref{fig:tower-ex}, we sketch a small Tower example that is
representative of a device driver that blinks an LED.  Small simplifications to
Tower have been made in the code, eliding details relating to code generation
and backend selection.

In the first column of the figure, the communication
architecture is defined in the \cd{Tower} monad.  The program initializes a
unidirectional channel between two tasks as well as the tasks themselves.  A channel, or queue,
consists of transmit (\cd{tx}) and receive (\cd{rx}) endpoints, respectively.
The \cd{blinkTask} task is an RTOS task that will send output to the
\cd{lightswitch} RTOS task via an RTOS-mediated channel.  The \cd{lightswitch}
task toggles the LED based on the incoming Boolean values.  (In the third column, a graph of the tower program is shown, generated from the
Tower compiler's Graphviz dot output, showing the architectural structure of the
two tasks as well as the queue between them.)

To conserve space, we only define \cd{blinkTask}.  The second column contains
the definition of \cd{blinkTask}, defined in the \cd{Task} monad.  The
\cd{blinkTask} task takes a channel source and returns a task.  The
task first initializes an \emph{emitter} for the channel then creates a
reference to allocated memory that is private to the task.  Every 100
milliseconds, an Ivory action is taken.  In this case, the action is to call
Ivory function \cd{blinkfromTime} that is executed whenever the task is enabled
(we elide the implementation of \cd{blinkfromTime} in this example).  The
boolean value \cd{res} is then emitted on the channel.


%% . The \cd{blinkTask} task is has fields for
%% properties and communications ports, showing its periodic rate and emitter. The
%% \cd{lightswitch} task only has one port, the event handler created by the
%% \cd{onChannel} primitive. A channel, annotated with its Ivory datatype
%% \cd{IBool}, is shown as a directed graph edge connecting from \cd{blinkTask}'s
%% emitter to \cd{lightswitch}'s event handler.




\section{SMACCMPilot: a High-Assurance Autopilot}

\begin{itemize}
	\item what is smaccmpilot?
		\begin{itemize}
			\item flight control software for small quadcopters
			\item bare metal on a microcontroller based board
			\item responsible for reading sensors, computing control
				outputs, and writing outputs to motor
				controllers in real time, at 200hz
			\item encrypted link for command, control, and telemetry
		\end{itemize}
	\item size/complexity of system
		\begin{itemize}
			\item tower complexity
				\begin{itemize}
					\item tasks
					\item channels
					\item in many cases, we use more tasks
						than are strictly required,
						mostly because individual event
						loops \& channels are a nice
						method of building abstractions
				\end{itemize}
			\item haskell loc: 10kloc application code, 3kloc bsp,
				10kloc generated from MAVLink message schema (an
				IDL with standard binary representation)
			\item c loc: generates 48kloc, depend on external
				libraries for another 10kloc, + os
			\item comparison to existing systems
				\begin{itemize}
					\item apm project
						60kloc code, supports multiple
						platforms
					\item px4 project
						25kloc application code
						25kloc support code, + big os
					\item these support a higher amount of
						autonomy than smaccmpilot, but
						the basics - IO, sensor fusion,
						flight stabilizer, telemetry,
						command \& control, make up the
						majority of the code.
				\end{itemize}
		\end{itemize}
\end{itemize}


%% \section{Ivory \& Tower Example Code}
\label{sec:examples}

\paragraph{Ivory Language Interfaces}
Ivory has different ``inferfaces'' for writing programs, defining procedures to
encapsulate programs (and that are compiled to C functions), defining C-level data
structures, and for constructing modules.  We cannot cover all of these here, so
we focus on the most central, explaining them by way of an example.

\begin{figure}
    \begin{smcode}
[ivory|
struct foo\_struct
  \{ bar :: Stored Uint8
  ; baz :: Array 10 (Stored Sint16)
  \}
|]

foo\_init :: Def(`[Ref s (Struct "foo\_struct")]:-> ())
foo\_init = proc "foo\_init" $ \textbackslash{}f\_ref -> body (prgm f\_ref)

prgm :: Ref s (Struct "foo_struct") -> Ivory eff ()
prgm f_ref = do
  store  (f_ref \(\sim\)> bar)      1
  store ((f_ref \(\sim\)> baz) ! 0) 2

foo\_module :: Module
foo\_module = package "foo\_module" $ do
  defStruct (Proxy :: Proxy "foo_struct")
  incl foo\_init
    \end{smcode}
  \caption{Example Ivory module definition}
  \label{fig:module}
\end{figure}
%% $

Consider Figure~\ref{fig:module}.  First, we define a struct (or product type)
using a quasiquoter.  Struct \cd{foo\_struct} contains two fields
consisting of an unsigned byte and an array of 10 signed 16-bit integers.  The
\cd{Stored} type constructor signifies that the value is allocated
in-memory~\cite{memareas}.  The \cd{Array} type constructor takes a type-level
natural number as a parameter (implemented as a Glasgow Haskell Compiler
extension) to fix the size of an array.

The quasiquoter uses Template Haskell~\cite{th} to generate
typeclass instances for the defined struct as well as top-level values that are
field accessors.  For example, for the \cd{bar} field, a label
\begin{code}
bar :: Label "foo\_struct" (Stored Uint8),
bar = Label "bar"
\end{code}
\noindent
is generated.  We describe the use of labels shortly.

A procedure, corresponding to a C function, has a type of the form
\cd{Def (params :-> out)}
where \cd{params} are the procedure's parameter types and \cd{out} is its return
type.  The procedure \cd{foo\_init} has a single input and its return type is
unit, corresponding to the \cd{void} type in C.  The types of the procedure's
inputs are types in a type-level list (the (\cd{`}) promotes the list value
constructor to a type constructor~\cite{datakinds}).  The type
\cd{Ref s area}
is the type of a \emph{reference}, or pointer guaranteed to be non-null by
construction.  A reference type constructor takes a \emph{scope} type and a
memory-area type.  The scope type denotes stack-allocated scope, or global and
statically allocated scope.  In the example, the parameter to \cd{foo\_init} can
be a reference to either scope.  The memory area pointed to be the reference
has the type of the struct defined above.

Now we describe the procedure \cd{foo\_init}.  The function \cd{proc} takes a
string corresponding to the name of the function that will be generated in C,
then a function from the procedures arguments to its body.  The lambda-bound
variable \cd{f\_ref}'s type is
\begin{code}
Ref s (Struct "foo\_struct")
\end{code}
\noindent
Before the body of a procedure, the user may add expression-level preconditions
on inputs and postconditions on return values. (These are not shown in the
current example).

The body of the procedure is defined by \cd{prgm}.  Its return type is an Ivory
expression, returning unit.  The Ivory monad also contains an effects parameter
that tracks effects, as described above, using a type-level record.

\newcommand{\mytilde}{\raise.17ex\hbox{$\scriptstyle\mathtt{\sim}$}}

Two statements implement the program, both of which update memory areas.  The
\cd{store} operator takes a reference and a value, and stores the value in the
the memory area pointed to by the reference.  The struct accessor operator,
reminiscent of C's \cd{->} operator, has the type
\begin{code}
(\(\sim\)>) :: Ref s (Struct symbol)
      -> Label symbol field -> Ref s field
\end{code}
\noindent
again eliding type class and data-kind constraints.  Note, however, that
gives \cd{\mytilde{}>} gives a reference to the field, not the field value.

The \cd{bar} field of the struct pointed to is updated with 1.  Updating
\cd{baz} is slightly more complicated since it contains an array. In
the example, the 0th index of the array in the \cd{baz} field is updated with 2
(the other indexes are unmodified).

Array indexing is guaranteed to be safe by the type system.  Like arrays,
indexes into arrays have types that are parameterized by a type-level natural
number.  An index type \cd{Ix n} only supports indexes from \cd{0} to \cd{n-1}.
Eliding type-class and data-kind constraints for the sake of presentation, the
type of the index operator ensures that the array length and index size match:
\begin{code}
(!) :: Ref s (Array len area) -> Ix len -> Ref s area
\end{code}

The final interface we describe in this example if our module interface.
A \cd{Module} contains all of the top-level procedures and struct definitions to
be sent to the compiler.  The module \cd{foo\_module} contains the definition
for the struct and the implementation of the function.  Because structure names
are type-level strings which are not \cd{*}-kinded, a proxy type is given to
turn an arbitrary-kinded type into a \cd{*}-kinded type (which can be passed to
a function). We discuss drawbacks of this module system design in
Section~\ref{sec:thegood}.

\begin{figure*}
  \begin{tabular}{p{0.36\textwidth}|p{0.30\textwidth}|p{0.33\textwidth}}
    \begin{smcode}
func :: Integer
     -> Def ('[Ref s (Stored IBool), Uint32]
             :-> ())
func period = proc "func" $ \textbackslash{}res currTime ->
  body $ do
    even <- assign (currTime .% (2 * p) <? p)
    store res even
    where
    p = fromIntegral period
    \end{smcode} &
    \begin{smcode}
blink :: ChannelSource (Stored IBool)
      -> Task ()
blink chan = do
  tx  <- withChannelEmitter chan
  res <- taskLocal
  onPeriod period $ \textbackslash{}now -> do
    call_ (func period) res now
    emit_ tx res
  where period = 100 :: Integer
    \end{smcode} &
% $
    \begin{smcode}
blinkApp :: Tower ()
blinkApp = do
  (tx,rx) <- channel
  task "blink" (blink tx)
  task "lightswitch" $
    onChannel rx $
      \textbackslash{}lit -> do
        ifte_ lit (turnOn light)
                  (turnOff light)
    \end{smcode}
  \end{tabular}
  \caption{Ivory (Column 1) and Tower (columns 2-3)}
  \label{fig:tower-ex}
\end{figure*}
\paragraph{Tower example}
In Figure~\ref{fig:tower-ex}, we sketch a small Tower example that is
representative of a device driver that blinks an LED at the rate of 100 ticks
(the duration of a tick is dependent on the RTOS backend to Tower used.)  Small
simplifications to Tower have been made in the code, eliding details relating to
code generation and backend selection.

Consider the first column of Figure~\ref{fig:tower-ex}.  The function takes an
integer and returns an Ivory procedure, which compiles to a C function.  The
procedure takes two arguments, a reference to a Boolean and a unsigned 32-bit
value.  The procedure creates a local variable named \cd{even},
assigning to it the value of an arithmetic expression.  The expression
takes a current time, which is a multiple of the period, and returns whether the
current time divided by the period is even or odd.  (Operators have a preceding
\cd{.} or a following \cd{?} for Boolean operators to avoid name-space collision
with Haskell operators.)  The result of the expression is stored into the
\cd{res} reference.

The \cd{blink} task is defined in column two.  It takes a channel source and
returns a program in the \cd{Task} monad.  The task first
initializes an emitter for the channel then creates a reference to private but
globally-allocated memory.  Every 100 ticks, an Ivory action is taken.  In this
case, the action is to call an Ivory function, the one described in the
previous section, that toggles that value pointed to by \cd{res}.  This value
pointed to by \cd{res} is then emitted on the channel.

In the third column the program, defined in the \cd{Tower} monad, initializes a
channel between two tasks as well as the tasks themselves.  A channel, or queue,
consists of transmit (\cd{tx}) and receive (\cd{rx}) endpoints, respectively.
The \cd{blink} task is an RTOS task that will send output to the
\cd{lightswitch} task, which toggles the LED based on the incoming Boolean
values.



\section{The Good}

In this section, we discuss some of the benefits of EDSLs for embedded
programming.  Some of these benefits were surprising to us, even with our
previous experience in functional programming and embedded development.

\paragraph{Type-checking as interpretation}
Build times are non-trivial for large software systems.  (Some of our kernel
developers at Galois are known to have novels on hand to read during the build
and test cycle.)  Modern C compilers are reasonably fast, but compiling 10s of
thousands of lines of code can take 10s of seconds.  In addition, during testing
and integration, our build includes various test builds and builds for multiple
operating-system and hardware configurations meaning that the same source code
gets recompiled multiple times.

Then, to execute the software on the embedded device, we have to write the
software to the device's memory (we refer to this as ``flashing'' the device,
since the executable is stored in non-volatile flash memory on a number of
microcontrollers).  Flashing can take many seconds.

All this is to say that the time between making a change in the sources to when
the changes are tested can be many seconds or minutes.  The situation is
exacerbated when we use an EDSL to generate the C code; the EDSL must be
compiled by Haskell; the haskell program is executed to generate the C code, and
then the C code is compiled, linked, and flashed.

High-level languages like Haskell have a read-eval-print-loop (REPL),
significantly reducing development time since it allows the developer to
type-check and test programs without going through the full compilation cycle.
Embedded C does not have a REPL, and building one is a major undertaking: it
would require a model of the sensors and devices the program interacts with, a
model of the operating system, as well as library code (e.g.,
glibc).\footnote{An emulator like QEMU~\cite{} does not solve the problem: (1)
  QEMU is not an interpreter, and often it is not useful at all if the
  application code interacts with sensors and external devices.}



\paragraph{Mario programming}
We are all just plumbers.

\paragraph{The five-minute driver}
Safe drivers as easy as user-code.  (Note Microsoft focus on drivers)

\lp{no seg faults}




%% \section{The Bad}


\begin{itemize}
\item type-checking errors, compile-cycle (Haskell + C). 
\item Tracking file/line no. from C to haskell.
\item Code duplication.
\item macro-expansion (i.e., memory addresses).
\item cabal with make
\end{itemize}




\section{Conclusions}
\label{sec:conclusions}

We have described our use of the Ivory and Tower EDSLs for building a large
embedded system. 

Many of the advantages of EDSLs for embedded programming relate to type-checking
in Haskell.  Of course, some bugs cannot be caught statically.  For the most part,
once type-checking is complete, we are confident that the bug is a logical bug.
We do not spend our time chasing segmentation faults or strange undefined or
compiler-dependent behaviors but rather focus on the bugs result from our
misunderstanding of the application, not the programming environment.

What is next?  In the next few years, SMACCMPilot will continue to grow. It,
along with the Ivory \& Tower tools, are open source, in the hope of engaging a
broader community.
We will add new hardware, new sensors, and new controllers so that it is not
only one of the highest-assurance autopilots in existence but is competitive
with others in terms of functionality.

In addition, we are looking to improve the usability of Ivory and Tower.  For
example, we are working to integrate verification tools more closely into the
language.  We have also begun to define quasiquoters for the languages, so that
C programmers might feel more at home with the language but power (Haskell)
users can still enjoy the benefits of EDSL programming.

In short, we believe EDSLs can be brought down from the ivory tower (pun
intended) to the grungy world of embedded programming.









%% \appendix
%% \section{Appendix Title}



\acks

This work is supported by DARPA under contract no. FA8750-12-9-0169.  Opinions
expressed herein are our own.  A number of people have provided input and
advice; we particularly thank Kathleen Fisher, Iavor Diatchki, and Andrew
Tridgell.  Joe Kiniry and Adam Foltzer, and the anonymous reviewers provided
helpful comments on earlier drafts of the paper.

% We recommend abbrvnat bibliography style.

\bibliographystyle{abbrvnat}
\bibliography{paper}

% The bibliography should be embedded for final submission.

%% \begin{thebibliography}{}
%% \softraggedright

%% \bibitem[Smith et~al.(2009)Smith, Jones]{smith02}
%% P. Q. Smith, and X. Y. Jones. ...reference text...

%% \end{thebibliography}


\end{document}

%                       Revision History
%                       -------- -------
%  Date         Person  Ver.    Change
%  ----         ------  ----    ------

%  2013.06.29   TU      0.1--4  comments on permission/copyright notices

