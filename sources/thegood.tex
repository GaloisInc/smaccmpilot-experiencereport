\section{Lessons Learned}
\label{sec:thegood}

In this section, we discuss some of the benefits and challenges of
using EDSLs for embedded programming, focusing on those that were surprising to
us, despite our teams' previous experience in functional programming and
embedded development.

Our experience using Ivory and Tower to build SMACCMPilot has been an extreme
lesson in ``eating our own dog-food'' EDSLs.  We had multiple developers writing the compilers
and using them to build applications concurrently.  We learned a few lessons
that are relevant to any compiler development but particularly relevant to EDSL
development.

\paragraph{Type-checking for embedded programming}
Build times are non-trivial for large software systems.  At the time of writing,
a fresh build of SMACCMPilot and associated test programs is over seven minutes
of real time (and 12~minutes of CPU time since we have a multi-threaded build
system).  One reason the build time is so large is that it requires Cabal (the
Haskell package manager) to discover library dependencies and install packages,
compile the Haskell sources, and then compile the C sources.  As well, some
sources are compiled multiple times for different targets on
multiple operating systems.

Then, to execute the software on the embedded device, we have to write the
software to the device's memory via a JTAG programmer or a serial bootloader,
which takes on the order of ten seconds.

All this is to say that the end-to-end debug cycle might mean testing a small
number of changes to Ivory or Tower per hour.  Clearly, the debug cycle in
embedded development particularly motivates us to make fewer bugs and to
discover them early.

%% Compilers for high-level languages like Haskell, such as GHC, generally have a
%% read-eval-print-loop (REPL), significantly reducing development time since it
%% allows the developer to type-check and test programs without going through the
%% full compilation cycle.

%% C compilers do not have an interpreter, and building one, particularly for
%% embedded hardware, would be a major undertaking requiring a model of the
%% sensors and devices the program interacts with, a model of the operating
%% system, as well as library code (e.g., glibc).  In short, we are stuck with
%% doing most testing on the real hardware.\footnote{An emulator like
%% QEMU~\cite{} does not solve the problem: (1) QEMU is not an interpreter, and
%% often it is not useful at all if the application code interacts with sensors
%% and external devices.}  For similar reasons, we do not generally use an
%% interpreter with Ivory (although we have built one).

During development, it became apparent how useful Haskell type-checking
is for embedded programming.  As described in Section~\ref{sec:ivory}, we
have embedded Ivory's type system in Haskell's.  Thus, domain-specific
type-errors are caught during Haskell type-checking.  Type-checking, and other
static warnings reported by GHC, are nearly instantaneous
since it can be done on a module-by-module basis.  The upshot is that Ivory
programs that would generate unsafe C programs are caught immediately.
The type system tracks the global or stack frame provenance of
references, as well as structure accessors and array indices, to ensure
all well-typed Ivory programs generate memory-safe C.

In addition, we have found it useful to detect potential bugs even if the C
compiler might also detect them.  To take one example, consider unused variable
declarations.  While a C compiler can detect this, perhaps late in the
compilation phase, we discover these warnings nearly instantaneously during type
checking.  Moreover, the more preprocessing we can do in Haskell, the more
potential errors we may find, and with a better relation to the source.

Not every property of interest in embedded programming can be conveniently
embedded in the Haskell type system, with GHC extensions.  For example, integer
overflows checks are not practical to embed.

Moreover, GHC's type error reporting can be unwieldy. Ivory users would benefit
from domain-specific error reporting which could, for example, describe type
errors in the vocabulary of Ivory, rather than burden the user to interpret the
way the Ivory language types are embedded into Haskell types.  For example,
passing the wrong number of arguments to a an Ivory function in a procedure call
is reported as a type error when using functional dependencies, and a mismatch
between the type of a procedure and the number of arguments provided in its
declaration is reported as a kind error.  The errors reported are of the
particular type-level implementation given for Ivory types.  Haskell does not
have good facilities for type-level programming abstraction.

\paragraph{Type-safe system plumbing}
The most tedious part of adding many new features to SMACCMPilot is the business
logic in Tower: defining a new task, and then plumbing all the communication
channels through the code.  There is nothing conceptually difficult in doing so.
It simply requires modifying the arguments to a Haskell function that generates
a Tower task (or modifying the fields of a data-type if channels have been
grouped together).  Channels are typed, so type-checking detects most plausible
inter-task communication errors.

Stepping back, the idea that plumbing arguments to Haskell functions is the
hardest part of embedded development is amazing.  We are not dealing with bugs
in low-level OS interfaces, we are not making timing errors in communication, we
are not dealing with type-errors like you might find in raw C (where data might
be cast to \cd{void*} or \cd{char[]}).

Because plumbing is so easy, it encourages us to improve modularity in the
system.  Defining a new RTOS task is easy, so we might as well modularize
functionality to improve isolation and security.  For example, in the ground
station communication subsystem, encryption and decryption are each executed in
isolated tasks, simplifying the architectural analysis of the system.  As
we noted in Section~\ref{sec:smaccmpilot}, our system is significantly more
modular than other autopilots.

\paragraph{Faking a module system}
In Ivory and Tower, top-level functions and structures are packaged into a
Haskell data structure to provide to the Ivory compiler.  The onus is on the
programmer to package up all the necessary components.

On one hand, the approach provides the programmer control over how to
modularize the generated C code, deciding which definitions to put in a C
source or header file.  On the other hand, we have found it to be verbose,
tedious, and error-prone.  Generally, we want the C files to have similar
structure to the Haskell modules in which Ivory programs are written.  From that
respect, the Ivory module system simply duplicates the Haskell module system.
We found that needing to manage the module system by hand, rather than
automatically in the compiler, was the biggest drawback to embedding Ivory in
Haskell.

Worse, moreover, is when the programmer forgets to package a definition.  The
error only becomes apparent at C \emph{link} time, near the end of a long build
process.  Missing definitions have plagued our builds.

We could move symbol resolution up the build cycle to the C-code generation
phase.  Ideally, we would move it up the build cycle even further.  We are
currently exploring the use of Template Haskell to generate Ivory modules at
compile-time to assist the programmer.

\paragraph{Control your compiler}
If we were writing our application in a typical compiled language, even a
high-level one, and found a compiler bug, we would perhaps file a bug report
with the developers... and wait.  If we had access to the sources, we might try
making a change, but doing so risks introducing new bugs or at the least, forking
the compiler.  Most likely, the compiler would not change, and we would either
make some \emph{ad-hoc} work-around or introduce regression tests to make sure
that the specific bug found is not hit again.  Such a situation is notorious in
embedded cross-compilers that usually have a small support team and are
themselves many revisions behind the main compiler tool-chain.

But with an EDSL the situation is different.  With a small code-base
implementing the compiler, it is easy to write new passes or inspect passes for
errors.  Rebuilding the compiler takes seconds.

More generally, we have a different mindset programming in an EDSL: if a class
of bug occurs a few times---whether caused in the compiler or not---we change
the language/compiler to eliminate it (or at least to automatically insert
assertions to check for it).  Instead of a growing test-suite, we
have a growing set of checks in the compiler, to help eliminate both our bugs
and the bugs in \emph{all} future Ivory programs.

\paragraph{Everything is a library}

\begin{figure}
  \begin{tabular}{p{0.5\columnwidth}|p{0.5\columnwidth}}
    \begin{smcode}
data Cond eff a =
  Cond IBool (Ivory eff a)

(==>) :: IBool -> Ivory eff a
      -> Cond eff a
(==>) = Cond

cond_ :: [Cond eff ()]
      -> Ivory eff ()
cond_ [] = return ()
cond_ ((Cond b f):cs) =
  ifte_ b f (cond_ cs)
    \end{smcode} &
    \begin{smcode}
cond_
  [ x >? 100 ==> ret 10
  , x >? 50  ==> ret 5
  , true     ==> ret 0 ]

ifte_ (x >? 100)
 (ret 10)
 (ifte_ (x >? 50)
   (ret 5)
     (ret 0))
    \end{smcode}
  \end{tabular}
  \caption{Conditional Ivory macro.}
  \label{fig:ivory-cond}
\end{figure}

With an EDSL, and particularly a Turing-complete macro language, everything is a
library.  The distinction between language developers and users becomes
ambiguous.  As an extreme example, one can think of Tower as ``just'' a library
for Ivory.  A small example is defining a conditional operator in terms of
Ivory's if-then-else primitive as shown in Figure~\ref{fig:ivory-cond}.  All
types above were introduced in Section~\ref{sec:ivory}.  With the \cd{cond\_}
operator, we can replace nested if-then-else statements as shown in the figure
with more convenient conditionals, without modifying the language.

Because macros are so easy to define and natural in EDSL development, our
biggest challenge is ensuring developers on our team put useful ones in a
standard library, to be shared.

%% \paragraph{Simplification is necessary}
%% \lp{not sure if this is that interesting}
%% We wrote a constant folder for Ivory as a ``nice to have'' feature.  

\paragraph{Compiler assurance}

We claim that Ivory code compiles to memory-safe C code.  However, a formal
proof of these claims, or more generally, a proof that the semantics of Ivory
programs are implemented by the generated C code, is work-in-progress.  However,
a small number of primitives and simple compiler facilitates compiler inspection
and testing. In all, this is less assurance than is given by fully verified
toolchains, such as CompCert~\cite{compcert}.  Other approaches more specific to
brining assurance to EDSL compilers~\cite{pike-icfp-12} could be borrowed as
well.

\paragraph{Semantics}
To take advantage of legacy cross-compilers, we are forced to
generate C code from our EDSL.  A large focus in designing Ivory is to allow
expressive but well-defined programs.  We believe Ivory
cannot produce memory-unsafe C programs.  However, undefined C programs can be
generated from Ivory; for example, signed integer overflow and division-by-zero
are undefined.  Guaranteeing programs are free from these behaviors is decidable
(the arithmetic is on fixed-width integer types), but intractable to prove
automatically.

To assist the programmer, the Ivory compiler automatically inserts predicates
into the generated code to check for overflow, division-by-zero, etc.  The user
defines the behavior of the program if a check fails.  For example, during
testing, we define the checks to insert a breakpoint for use with a debugger.
Another option may be to do nothing and rely on the semantics provided by the C
compiler.  Still another option might be to trap to a user-defined
exception-handler.  Currently, SMACCMPilot contains approximately 2500
compiler-inserted non-trivial checks that cannot be constant-folded away.
In the future, we hope to \emph{prove} these checks never fail.

Early in the
development process, we used the CBMC model checker~\cite{cbmc} to partially
verify the assertions in the generated C~code. However, as our
application grew, we ran into three problems. First, a naive application
of whole program model checking did not scale to our application size. Second,
many assertions depend on user-provided preconditions (e.g. on inputs from
hardware devices). Third, some assertions were undecidable (e.g. non-linear
arithmetic).

There are two other semantics categories to consider: defined behavior and
implementation-defined behavior.  In Ivory, we attempt to eliminate almost all
implementation-defined behaviors.  For example, only fixed-width size types,
like \cd{uint8\_t} or \cd{int32\_t}, can be generated.  Implementation-defined
sizes, like \cd{int} or \cd{char} are not used.  We have found these to be
dangerous: programmers might assume properties about the size of a type that do not
hold in a non-standard architecture (e.g., that an \cd{int} is at least 32 bits
or that \cd{char} is unsigned; both are implementation-defined).  Such
assumptions are particularly dangerous when porting code between different
embedded platforms.  Indeed, when we ported portions of ArduPilot, initially built
for an 8-bit AVR architecture to a 32-bit ARM, we found these sort of implicit
assumptions.

Finally, even defined behavior is not necessarily intuitive behavior.  For
example, in C, the defined behavior for arithmetic on values that have a
size-type smaller than \cd{int} is to implicitly promote them to \cd{int}s before
performing the arithmetic.

For example, given
\begin{code}
uint8\_t a = 10;
uint8\_t b = 250;
bool    x = a-b > 0;
bool    y = (uint8\_t)(a-b) > 0;
\end{code}
\noindent
\cd{x} evaluates to 0 and \cd{y} to 1, provided that
\begin{code}
sizeof(uint8\_t) < sizeof(int)
\end{code}
\noindent
This behavior is important to the embedded programmer because, across various
embedded processors and C compilers, integer sizes are often defined
differently.

In Ivory, arithmetic is at the size of the operands, which we believe is more
intuitive.  We force the generated C to respect this semantics by inserting
casts into expressions.  So the Ivory expression \cd{a-b} results in the C
expression \cd{(uint8\_t) (a-b)}.

%% \paragraph{Syntax}
%% Syntax matters.  The importance of syntax is not just a matter of familiarity to
%% the user can help make the language better.

%% We have used Haskell's quasiquoter already to define new syntax
%% for defining structs.  For example,
%% \begin{code}
%% [ivory| struct foo \{ a :: Stored Uint32
%%                    ; b :: Array 10 (Stored Sint32)
%%                    \} |]
%% \end{code}
%% \noindent
%% is a quasiquoter that defines a struct \cd{foo} with two fields.  The first is
%% an unsigned 32-bit integer and the second is an array of 10 signed 32-bit
%% integers.  (The \cd{Stored} type constructor tells us that the values are
%% allocated in memory.)  The quasiquoter automatically generates selector
%% functions that return each field of the struct.  We have a quasiquoter to define
%% bit-fields~\cite{high-level}.

%% We are in the process of defining a quasiquoter for other portions of the
%% language.  With a quasiquoter for the statement and expression language, awkward
%% monadic constructions can be written more simply.  For example, in Ivory,
%% dereferencing a pointer is a monadic action and so is a statement.  However, it
%% is pure action, so we can treat it as an expression.  Compare

%% \lp{finish.  Discuss issues with module system, reflection of function names.
%%   Type-checking after parsing.}

%% \begin{code}
%% val <- deref (arr ! 5)
%% store (arr ! 5) (val + 1)
%% \end{code}

%% \begin{code}
%% arr[5] = arr[5] + 1;
%% \end{code}

%% \paragraph{Safe drivers}
%% \ph{worth having a section like this?,
%%     tower plus bit-data makes drivers easy.
%%     If so, can you write this?
%%  -- Actually, I don't think this is as valuable as other sections here
%% }

%% Safe drivers as easy as user-code.  (Note Microsoft focus on drivers)

%% A sports science research article has recently popularized the idea of a
%% ``7~minute workout'' that is short but intense~\cite{}.






