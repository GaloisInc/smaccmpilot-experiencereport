\section{Conclusions}
\label{sec:conclusions}

We have described our use of the Ivory and Tower EDSLs for building a large
embedded system.

Many of the advantages of EDSLs for embedded programming relate to type-checking
in Haskell.  Of course, some bugs cannot be caught statically.  For the most part,
once type-checking is complete, we are confident that the bug is a logical bug.
We do not spend our time chasing segmentation faults or strange undefined or
compiler-dependent behaviors but rather focus on the bugs result from our
misunderstanding of the application, not the programming environment.

What is next?  In the next few years, SMACCMPilot will continue to grow. It,
along with the Ivory \& Tower tools, are open source, in the hope of engaging a
broader community.
We will add new hardware, new sensors, and new controllers so that it is not
only one of the highest-assurance autopilots in existence but is competitive
with others in terms of functionality.

In addition, we are looking to improve the usability of Ivory and Tower.  For
example, we are working to integrate verification tools more closely into the
language.  We have also begun to define quasiquoters for the languages so that
C programmers might feel more at home with the language but power (Haskell)
users can still enjoy the benefits of EDSL programming.

In short, we believe EDSLs can be brought down from the ivory tower (pun
intended) to the grungy world of embedded programming.






