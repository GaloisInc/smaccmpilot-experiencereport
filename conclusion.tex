\section{Conclusions}

\lp{talk about assurance/productivity trade-offs}

\lp{quasi quotation}



\paragraph{Dog-fooding}
Our experience using Ivory and Tower to build SMACCMPilot has been an exteme
lesson in dog-fooding EDSLs.  We had multiple developers writing the compilers
and using them to build applications concurrently.  There were real-time
dependencies in our development.  For example, some of the authors (who shall
remain nameless!)  introduced compiler bugs discovered by application
developers.  The compiler bug had to be fixed in real-time.

We learned a few lessons that are relevant to dog-fooding any compiler
development but particularly relevant to EDSL development giving the fast
development time.

\begin{itemize}
  \item \emph{Everything is a library}.  With an EDSL, and particularly a
    Turing-complete macro language, everything is a library.  The distinction
    between language developers and users becomes ambiguous.  As an extreme
    example, one can think of Tower as ``just'' a library for Ivory.  A smaller
    example is defining a conditional operator in terms of the Ivory
    if-then-else (\cd{ifte\_}) primitive.  The full definition of \cd{cond\_} is below:
    \begin{code}
data Cond eff a = Cond IBool (Ivory eff a)

(==>) :: IBool -> Ivory eff a -> Cond eff a
(==>) = Cond

cond_ :: [Cond eff ()] -> Ivory eff ()
cond_ [] = return ()
cond_ ((Cond b f):cs) = ifte_ b f (cond_ cs)
    \end{code}
\noindent
All types above were introduced in Section~\cite{sec:ivory} with the exception
of \cd{IBool}, which is the type of a Boolean, in Ivory.  With the \cd{cond\_}
operator, we can write conditionals like
\begin{code}
cond_ [ x >? 100 ==> store result 10
      , x >? 50  ==> store result 5
      , true     ==> store result 0 ]
\end{code}
\noindent
instead of
\begin{code}
 ifte_ (x >? 100)
   (store result 10)
   (ifte_ (x >? 50)
     (store result 5)
       (store result 0))
\end{code}

  \item \emph{Simplification is necessary}.
We wrote a constant folder for Ivory as a ``nice to have''

  \item \emph{Clear semantics}.
\end{itemize}

\paragraph{Pain-points}



