\section{Introduction}

Embedded programming involves the lowest levels of abstraction.  Most development
is in low-level languages, like C or assembly, and programs interact intimately
with the hardware.  Embedded domain-specific languages (EDSLs) are in
some sense at the other end of the software spectrum: they are often embedded
(in a different sense of the word!) in high-level programming languages such as
Haskell or ML, and are used to lift the abstraction level.

That said, there is no reason in-principle why EDSLs cannot be used for embedded
programming; this report is about our experience in building new EDSLs for
embedded programming and the benefits and difficulties in using them.  Our
experiences are based on building an autopilot system called SMACCMPilot from
our EDSLs.  The breadth and scope of the project sets it apart.  The autopilot
software is a complete embedded system that
includes not just the core flight control algorithms, but also device drivers,
encrypted network stack, mode logic, and concurrency and task management.
As far as we know, it is
one of the largest (open-source) embedded systems projects developed using the
EDSL approach.

Our story is a largely positive one: we developed the Ivory and Tower languages
and their (EDSL) compilers from scratch in approximately 14 engineer-months then
we used them to build the SMACCMPilot hardware support and application in
another 22 engineer-months.  We achieved a dramatic increase in productivity as
well as code quality.  By construction, the generated C excludes large classes
of errors and undefined behaviors.

Our goal in this paper is to summarize some of our lessons-learned.  While we
use specific examples, the lessons apply more generally to industrially-large
embedded system design in EDSLs.  Our target audience includes both researchers
developing new EDSLs for low-level programming as well as practitioners
considering using EDSLs.

All of the EDSLs described herein, as well as SMACCMPilot itself, are
open-source software. Further documentation is available at \url{smaccmpilot.org}.


%% \lp{Indeed, a number of researchers have explored EDSLs for generating embedded
%%   C code, such as Atom~cite{}, Feldspar~\cite{}, Copilot~\cite{}, \lp{more}.
%%   However, these languages are still at the application level }

%% \lp{In this experience report, we describe lessons learned in the use of EDSLs to
%% build a complex embedded system: a secure autopilot called
%% \emph{SMACCMPilot}.\footnote{\emph{SMACCM} is an acronym for Secure
%%   Mathematically-Assured Composition of Control Models.}
%% }

