\section{Introduction}

Embedded programming involves the lowest levels of software development.  Most
development is in a low-level language, like C or assembly, and programs
interact intimately with the hardware directly.  Embedded domain-specific
languages (EDSLs)~\cite{} are in some sense at the other end of the software
spectrum: they are often embedded (in a different sense of the word!) in
high-level programming languages such as Haskell or ML, and are used to
lift the abstraction level for programmers.

While the abstraction levels are quite different, there is no in principle
reason why EDSLs cannot be used for embedded programming; this report
is about our experience in building new EDSLs for embedded programming and the
benefits and difficulties in using them.  Our case-study is building a new
secure autopilot from the ground up.

The breadth and scope of the project sets it apart.  The term `autopilot'
actually understates the software system built; SMACCMPilot's use of EDSLs
includes not just the core flight control algorithms, but also device drivers,
encrypted network stack, mode logic, and concurrency and task management.  In
all, our project involved some \ph{XXX} lines of Haskell-hosted EDSL code that
generates \ph{YYY} lines of embedded C.  It is one of the largest and most
embedded system developed using the EDSL approach.

Our story is a largely positive one: we developed the languages and their (EDSL)
compilers from scratch in approximately \ph{XXX} engineer-months then we used
them to build SMACCMPilot in another \ph{YYY} engineer-months.  We achieved a
dramatic increase in productivity.  Furthermore, we achieved an increase in
code quality; our generated C code is free from memory-safety errors.

Our goal in this paper is to summarize some of our lessons-learned.  While we
use specific examples, the lessons apply more generally to industrially-large
embedded system design in EDSLs.  Our target audience includes both researchers
developing new EDSLs for low-level programming as well as practitioners
considering using EDSLs.


\lp{Indeed, a number of
researchers have explored EDSLs for generating embedded C code, such as
Atom~cite{}, Feldspar~\cite{}, Copilot~\cite{}, \lp{more}.  However, these
languages are still at the application level
}

\lp{In this experience report, we describe lessons learned in the use of EDSLs to
build a complex embedded system: a secure autopilot called
\emph{SMACCMPilot}.\footnote{\emph{SMACCM} is an acronym for Secure
  Mathematically-Assured Composition of Control Models.}
}

