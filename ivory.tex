\section{Ivory: Safe C Programming}

At face value: our approach sounds audacious if not simply ludicrous: faced with
developing new autopilot software in one-and-a-half years, start by designing a
new programming language and compiler from the ground up.

Of course, developing an EDSL is not the same as developing a stand-alone
compiler.  Much of the typical compiler tool-chain, like the front-end
parser/lexer, is obviated.  Furthermore, we were confident that time ``lost'' in
writing the EDSL would be made up by developing in a high-level functional
language with a rich set of existing libraries.

The language we developed for generating safe embedded C code is called
\emph{Ivory}.  Ivory shares the goal of other ``safe-C'' programming languages.
Notable related work includes \lp{Cyclone, Iavor's stuff, etc.}.  From a
practical standpoint, most of these languages have bit-rotted, but our main
motivation for not using them is that our desire for a Turing-complete,
type-safe macro-language to improve our productivity.

There have also been ``safe-C'' EDSL efforts.  Notable examples include
\lp{Atom, Copilot, FeldSpar}.  \lp{However, languages also have some
  drawbacks.}

The essential contributions of Ivory from a programming language perspective are
its expressiveness and our approach to type-safety.

\paragraph{Expressiveness}
Regarding the former, Ivory has a number of features:

\begin{compactitem}
  \item \emph{Memory-areas}: stack-based allocation of data~\cite{}.
  \item \emph{Product types}: C structs and well-typed getters.
  \item \emph{FFI}: typed interfaces for calling arbitrary C functions.
  \item \emph{Module system}: managing dependencies to support C-level headers and
    includes.
  \item \emph{Bit-fields}: support for typed manipulation of bit-fields and registers~\cite{}.
\end{compactitem}

As well, Ivory has some limitations, motivated by making analysis simpler to
ensure safe C programs are generated.  Ivory does not support heap-based dynamic
memory allocation (but global variables can be defined).  C arrays are
fixed-length.  There is no pointer arithmetic.  Pointers are non-nullable.
Union types are not supported.  Unsafe casts are not supported: casts must to a
strictly more expressive type (e.g., from an unsigned 8-bit integer to an
unsigned 16-bit integer) or a default value must be provided that is returned
for casts otherwise.  In particular, no void-type exists.

In Ivory, these have not been limiting factors, particularly because of the
power of using Haskell as a macro system.  For example, while arrays must be of
fixed size at C compile-time, we can define a single \emph{Haskell} function
that is polymorphic in the array size that becomes instantiated at a particular
size at each use site.  %% An array has the type

%% \begin{code}
%% Array (n::Nat) t
%% \end{code}

\paragraph{Type-level properties}
Ivory's domain-specific type checking focuses essentially on guaranteeing memory
safety and helping the programmer reason about her programs' nonfunctional
behaviors more easily.  With respect to memory safety, the essential
guarantees necessary, for our restricted form of C, are that
\begin{compactitem}
  \item a void pointer is not dereferenced;
  \item indexing into an array does not occur outside the memory allocated to
    it;
  \item no dangling pointers; in particular, a function should not return a
    pointer to memory allocated within its own stack frame.
\end{compactitem}

\noindent
Our claim is that for the fragment of C generated, these type checks are
sufficient to guarantee generated programs are memory safe (a forthcoming paper
will make this argument explicit).

In addition, Ivory programs have an \emph{effects} type associated with them.
There are three kinds of effects tracked:\footnote{These effects have no
  relation to the recently popularized monad effects work, e.g.,~\cite{}.}
\begin{compactitem}
  \item \emph{Allocation scope}: whether a program performs (stack-based)
    memory allocation as well as whether pointers point into global or stack memory.
  \item \emph{Returns}: whether a program contains a \cd{return} statement.
  \item \emph{Breaks}: whether a program contains a \cd{break}
    statement.
\end{compactitem}

\noindent
Tracking allocation effects helps in reasoning about memory usage, but the main
benefit is to reason about data provenance for pointers: pointers into a stack
frame cannot effect global memory.  Return and break statements fundamentally
effect control-flow and can result in unexpected behavior by breaking out of the
current block or returning from a function.  Such effects are particularly
dangerous in the context of an EDSL in which programs are generated and
manipulated heavily in the host language.

\subsection{Type checking}
In an EDSL, we have two options for type checking: (1) write a domain-specific
type-checker \emph{in} Haskell (relying on Haskell's type-system just for
macro-level type-checking), or (2) embed the domain-specific type checker into
Haskell's type checker.

For a large system such as SMACCMPilot, the development times are substantial.
Therefore, our goal is to learn about program errors early, motivating us to
pursue option (2).  (We discuss the issues of finding errors early on in more
detail in Section~\ref{}.)

Our hypothesis when developing Ivory was that the type-level extensions to the
version of Haskell supported by the Glasgow Haskell Compiler~\cite{} \lp{cite
  Hasochism Connor McBride paper} make it feasible to embed the type-checks
necessary to ensure memory-safe C programming into the type system.  From a
practical standpoint, doing so demonstrates just how far the type-system has
come, allowing us to replicate the type safety of compilers like Cyclone,
etc. without implementing our own type checker.

We do not have space to describe the full embedded type system for Ivory.  To
give the reader a sense of the type system, we show a portion of the effects
system and how dangling pointers are prevented in a toy version of the language.

\paragraph{Effects}
Using the data kinds~\cite{} and type families~\cite{} extensions, we define a
type-level record to track effects.  Using the data kind extension to lift data
types to define kinds and types, we define a kind \cd{RetEff} with the type
\cd{Ret r} denoting that an Ivory code block contains a return statement, and if
so, the type of the value returned.  If there is no return statement, the
associated type is \cd{NoRet}.
\begin{code}
data RetEff   = forall r. Ret r | NoRet
\end{code}
\noindent
We similarly define an allocation kind containing types that denote whether
there is local or global memory allocation as well as the type of the value
allocated:
\begin{code}
data AllocEff = forall s. Scope s | NoAlloc
\end{code}
\noindent
Using type families, we define rewrite rules that are type-level record
selectors that take a type of kind \cd{Effects} and returns the field (a type)
of interest:
\begin{code}
data Effects  = Effects AllocEff RetEff

type family   GetRet (effs :: Effects) :: RetEff
type instance GetRet ('Effects s r) = r

type family   GetAlloc (effs :: Effects) :: AllocEff
type instance GetAlloc ('Effects s r) = s
\end{code}
\noindent
(The tick mark disambiguates whether we use a term to denote a type or kind;
with the tick, it is used as a type.)  Our ``record approach'' allows us to
encode an arbitrary number of effects in a single type.  In Ivory, there is an
additional effect tracking the use of \cd{break} statements.

For convenience, we define a type synonym corresponding to effects for a C
function: it should contain a return statement and there may be memory
allocation:
\begin{code}
type FnEffects s r = 'Effects (Scope s) (Ret r)
\end{code}

\paragraph{Ivory Type Checking}
Ivory is a monadic language.  In demonstrating type-checking, the monad
implementation is not important except to note that it is parameterized by an
effects parameter.  (In practice, the Ivory monad is a writer transformer (to
record statements) over the state monad (to generate fresh names).  For our
purposes, here, we can use the identity monad.)
\begin{code}
newtype Ivory (eff :: Effects) a = Ivory ...
  deriving Monad
\end{code}
\noindent
Now we define a data type \cd{Body r} that contains the body of a Ivory
function.
\begin{code}
newtype Body r = Body
  { runBody :: forall s. Ivory (FnEffects s r) () }
\end{code}
\noindent
Its type parameter corresponds to the type of the value returned.  Using rank~2
quantification, we ensure the allocation type cannot escape.

We now have the machinery to define an Ivory statement.  \cd{ret} corresponds to
the \cd{return} statement in C:
\begin{code}
ret :: (GetRet eff ~ Ret r) => r -> Ivory eff ()
ret r = undefined
\end{code}
\noindent
The function \cd{ret} takes a value and returns a monadic action.  Note the
type-equality constraint that ensures that the \cd{Ret} field of the type-level
record equals \cd{Ret r}, corresponding to the value returned.  (Again, in this
model we do not care about the implementation of the function.)

Now we can write an Ivory program such as
\begin{code}
i0 :: Body Integer
i0 = Body (ret 42)
\end{code}
\noindent
Or to model returning \cd{void},
\begin{code}
i1 :: Body ()
i1 = Body (ret ())
\end{code}

Let us add a little more machinery to model allocation.  Again using the data
kinds extension, we define the kind \cd{RefScope} containing two types that
denote the scope of allocation.
\begin{code}
data RefScope
  = Global
  | forall s. Stack s
\end{code}
\noindent
A reference is a pointer that is guaranteed by construction to be non-null.  The
type constructor's first parameter is the scope of the reference and the second
is the type of the value referenced.  (In the actual implementation, \cd{a} is a
type with kind \cd{Area}, denoting all the types that can be allocated.)
\begin{code}
data Ref (ref :: RefScope) a = Ref
\end{code}
\noindent
In our model, we provide a trivial value constructor \cd{Ref}.


Now we can define an operator to allocate stack-based memory.
\begin{code}
local :: (GetAlloc eff ~ Scope s)
      => Ivory eff (Ref (Stack s) a)
local = undefined
\end{code}
\noindent
Just like with the \cd{ret} statement, there is a type equality constraint, but
the Ivory return type is a reference to a stack-allocated value.  The return
type, together with the rank-2 definition of \cd{Body}, ensures we cannot return
a stack-allocated reference created inside a function body.  (Again, we do not
care about the implementation.  Furthermore, in Ivory, \cd{local} takes an
initializer as an argument.

We need one more operator to dereference pointers.
\begin{code}
deref :: Ref ref a -> Ivory eff a
deref ref = undefined
\end{code}
\noindent
The \cd{deref} operator takes a reference and returns the referenced value.

Now we can define Ivory programs that allocate a local value, dereference it,
and then return the value, which is memory safe:
\begin{code}
i2 :: Body r
i2 = Body $ do
  l <- local
  v <- deref l
  ret v
\end{code}
%% $

We can also pass a stack-allocated reference to a function, work with it
(including storing new values in the memory area), and then return the reference:
\begin{code}
i2 :: Ref (Stack s) a -> Body (Ref (Stack s) a)
i2 l = Body $ do
  v <- deref l
  ...
  ret l
\end{code}
% $

However, returning a reference created locally is a type-error:
\begin{code}
i3 = Body (ret =<< local)
\end{code}
%% $
\noindent
No possible type can be given to \cd{i3}.  The ``closest'' type is something
like
\begin{code}
Body (Ref (Stack x) a)
\end{code}
\noindent
But the type variable \cd{x} must unify with the universally quantified
\cd{s} in the type of the argument to the value constructor \cd{Body}, which is
not possible.  Note that if the type of \cd{local} were instead
\begin{code}
(GetAlloc eff ~ Scope s) => Ivory eff (Ref Global a)
\end{code}
\noindent
The program \cd{i3} would be well-typed as \cd{Body (Ref Global a)}.




%% Memory-allocated types have a distinct kind.  The benefit of doing so is that
%% type classes with methods operating on memory areas are defined closed under .
%% The type of all data that can be allocated (either globally or locally) has kind
%% \cd{Area k}:
%% \begin{code}
%% data Area k
%%   = Array Nat (Area k)
%%   | Struct Symbol
%%   | ...
%% \end{code}
%% \noindent
%% We have elided a few additional memory area types.  Note to that the data type
%% is used only to define a kind and its types~\cite{}; we do not use \cd{Array},
%% etc. as value constructors.  The type constructor for an array type
%% takes a type-level natural number (from \cd{GHC.TypeLits}) representing its
%% length at the type-level and the type of the values contained in the array.
%% Similarly, the type constructor for a struct type takes a type-level string that
%% names the struct definition.


%% \begin{compactitem}
%% \item Brief overview of Ivory.
%% \item Note that types are embedded in GHC type-checker.
%% \item Philosophy: push errors higher into the design cycle (C runtime to C
%%   compile time to DSL runtime to typechecking time).
%% \item Insert checks into compiler.  Easy with EDSL  (blog post).
%% \end{compactitem}


%% \begin{compactenum}
%%   \item Haskell type-checking (Haskell compilation)
%%   \item Domain-specific type-checking
%%   \item C compilation
%%   \item Static analysis
%%   \item Execution
%% \end{compactenum}
